\section{Aufgabenstellung}
Für die Ballschussmaschiene Unihockey soll zusätzlich eine Geschwindigkeitserfassung entworfen werden.\\

Dazu gibt es folgende \marg{Randbedingungen}Randbedingungen:
\begin{itemize}
    \item  Messung der Ballgeschwindigkeit (max. 180km/h).
    \item Als Zusatz vor die Maschine zu hängen, Befestigung z.B. magnetisch.
    \item Eigenständige Einheit mit Arduino und Anzeige. 
    \item Bei Bedarf serielle Verbindung zum Master-Arduino möglich, z.B. um Richtungswinkel zu berücksichtigen.
    \item Sensorik: Z.B. zwei Lichtschranken in einem fixen Abstand, welche vom Ball unterbrochen werden. Evtl. führt man den Lichtstrahl mittels Spiegel mehrmals durch den Flugbereich, um den Einfluss der Flugbahnrichtung und der Kugelform des Balles zu reduzieren.
\end{itemize}

\subsection{Vorgehen}
Es \marg{Vorgehen} wurde folgendermassen vorgegangen:
\begin{itemize}
    \item Überblick über die bestehende Anlage verschafft.
    \item Mögliche Anordnung der Lichtschranken skizziert.
    \begin{itemize}
        \item Flugzeiten berechnet.
    \end{itemize}
    \item Kritische Parameter und mögliche Fehlerquellen identifiziert.
    \item Passende Lichtschranke ausgewählt und beschafft.
    \item Versuchsaufbau erstellt.
    \item Benötigte Software erstellt.
    \item Resultate validiert.
\end{itemize}

